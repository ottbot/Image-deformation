\documentclass[a4paper, 12pt]{article}


\usepackage{amsmath}
%\usepackage{vector}

\newcommand{\eq}[1]{Eq. \ref{eq:#1}}
\newcommand{\vect}[1]{\ensuremath{\mathbf{#1}}}
\newcommand{\hvect}[1]{\ensuremath{\hat{\vect{#1}}}}

\newcommand{\pp}[2]{\frac{\partial #1}{\partial #2}}

\setlength{\parskip}{.3cm}

\begin{document}


\begin{align*}
  \vect q(s) &= 0 \leq s \leq 2\pi \\
  \vect q(0) &= \vect q(2\pi)
\end{align*}


About the problem, and applications to computational anatomy...

Other approches to this.. LDDMM, EPDIFF, 

How our approach works..

Parameterisation invariant..
The  vector field is defined directly on the curve, rather than on the entire field (LDDMM) -- which deforms the space, rather resulting in a deformation of the space. Whereas we're deformng the shape only.


The details ---

The template $\vect q_A$ is the curve at $t=0$, and the target $\vect q_B$
is the curve at $t=1$. The scheme finds a velocity $\vect u(s)$ to minimize
distance between the two curves. A BFGS algorithm is used to find the minimal
functional $S[u]$,

a metric, such that the dq/ds ensures is parameterisation invariant -- and non zero!

\begin{equation}
  \label{eq:S}
  S = \frac{1}{2} \int^{1}_{t=0} \int^{2\pi}_{s=0}\left( \left| \vect u(s,t) \right|^2 
  \left| \frac{\partial \vect q}{\partial s} \right|  + 
  \alpha^2 \frac{ 
    \left| \frac{\partial \vect u}{\partial s}\right|^2}{
    \left| \frac{\partial \vect q}{\partial s}\right|}\right)  \,ds\,dt
  + \frac{1}{2\sigma^2}\int^{2\pi}_{s=0}\left| q(s,1) - q_B(s)\right|^2\,ds.
\end{equation}
Subject to
\begin{subequations}
\begin{equation}
  \vect q_A(s) =\vect q(s,0)   \label{eq:template}
\end{equation}
\begin{equation}
  \frac{\partial \vect q}{\partial t}(s,t) = \vect u(s,t)  \label{eq:dqdt}
\end{equation}
\end{subequations}
where $\alpha$ is dependent on the mesh size ????, and $\sigma$ is the weighing
of error term $\left| q(s,1) - q_B(s)\right|^2$.



Given a velocity $\vect u(s,t)$, the scheme first finds the vector $\vect q$ at
each time step from $t=0$ to $t=1$. \eq{dqdt} is discretized using a forward finite difference
\begin{equation}
  \label{eq:fd_q}
  \frac{\vect q(s)^{n+1}- \vect q(s)^n}{\Delta t} = \vect u(s).
\end{equation}
Using a test function $\vect r(s)$, \eq{df_q_test} is solved for $\vect q^{n+1}$
at each time step.
  \begin{equation}
    \label{eq:df_q_test}
    \int \vect{r}(s) \cdot \vect{q}^{n+1}(s)\, ds = \int \vect{r}(s) \cdot
    \vect{q}^n (s)\,ds + \Delta t \int \vect{r}(s)\cdot \vect{v}(s)\, ds
  \end{equation}



The variational derivative $\frac{\delta S}{\delta \vect u}$ is need to improve
the performance of the optimiser. 

For a single timestep, the variational derivative can be found with the equation
\begin{equation}
  \label{eq:dsdu}
  \langle \vect v, \frac{\delta S}{\delta \vect u}\rangle =
  \langle \vect v, \vect u \left| \frac{\partial \vect q}{\partial s}\right |\rangle 
  + \alpha^2 \langle \frac{\partial \vect v}{\partial s},
  \frac{\frac{\partial \vect v}{\partial s} }{ \left|\frac{\partial \vect q}{\partial s}\right|}
\rangle - \langle \vect v, \hvect{q} \rangle,
\end{equation}
where $\vect v$ is a test function, and $\hvect q$ is a Lagrange multiplier (SHOW..)

For each time step, $\hvect q$ is found using backwards finite difference method
where $\hvect q(s,1)$ is found by
\begin{equation}
  \label{eq:qh1}
  \langle \hvect p, \hvect q(s,1) \rangle = - \frac{1}{\sigma^2}
  \langle \hvect p, \vect q(s,1) - q_B \rangle
\end{equation}
where $\hvect p$ is a test function.


\begin{equation}
  \label{eq:dqhdt}
  \left\langle \hvect{p},\pp{\hvect{q}}{t}\right\rangle = \left\langle \frac{1}{2}\left(
      \frac{|\vect u|^2}{\left|\pp{\vect q}{s}\right|}
      -\alpha^2\frac{\left|\pp{\vect u}{s}\right|^2}{\left|\pp{\vect q}{s}\right|^3}\right)
    \pp{\hvect{p}}{s},\pp{\hvect{q}}{s}\right\rangle 
\end{equation}


  % \langle \hvect p, \frac{\partial \hvect q}{\partial t} \rangle =
  % \langle
  % \left( 
  %   \frac{1}{2}
  %   \frac{ 
  %     \left| \vect u \right|^2}{
  %     \left| \frac{\partial \vect q}{\partial s}\right|}
  %   +
  %   \frac{\alpha^2}{2}
  %   \frac{ 
  %     \left| \frac{\partial \vect u}{\partial s}\right|^2}{
  %     \left| \frac{\partial \vect q}{\partial s}\right|^2}
  % \right)\frac{\partial \hvect p}{\partial s}, \frac{\partial \hvect q}{\partial s} \rangle


Substituting $\frac{\partial \hvect q}{\partial t}$ with a backwards finite difference and
rearranging \eq{dqhdt},
\begin{equation}
  \label{eq:fd_qh}
  \int \hvect p \cdot \hvect q^n \,ds = \int \hvect p \cdot \hvect q^{n+1} \,ds
  - \int
   \left( 
      \frac{|\vect u|^2}{\left|\pp{\vect q}{s}\right|}
      -\alpha^2\frac{\left|\pp{\vect u}{s}\right|^2}{\left|\pp{\vect q}{s}\right|^3}
  \right) \frac{ \partial \hvect p}{\partial s} 
  \cdot \frac{\partial \hvect q^{n+1}}{\partial s}\,ds
\end{equation}
is solved for $q^n$ at each timestep.



\section{TODO/Ask}
\begin{itemize}
\item Use of fmin\_l\_bfgs\_b instead of fmin\_bfgs
\item Getting fmin\_bfgs to stop at the right spot, how to find the best value of $\sigma^2$?
\item Why is the sign for $\frac{\delta S}{\delta u}$ wrong in the code!?
\item Memory issues with fmin\_bfgs
\end{itemize}

\end{document}