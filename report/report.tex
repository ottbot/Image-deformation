\documentclass[a4paper, 10pt]{article}


\usepackage{amsmath}
\usepackage{vector}
\usepackage{cite}
\usepackage{graphicx}


%\usepackage{mathpazo}
%\usepackage[mathpazo]{flexisym}

%\usepackage{breqn}

\graphicspath{{./img/}}

% \pdfpagewidth=\paperwidth 
% \pdfpageheight=\paperheight

\newcommand{\eq}[1]{(\ref{eq:#1})}
\newcommand{\vect}[1]{\ensuremath{\mathbf{#1}}}
\newcommand{\hvect}[1]{\ensuremath{\hat{\vect{#1}}}}

\newcommand{\pp}[2]{\frac{\partial #1}{\partial #2}}
\newcommand{\vv}[2]{\frac{\delta #1}{\delta #2}}
\newcommand{\angles}[1]{\left\langle #1 \right\rangle}
\newcommand{\eps}{\varepsilon}
\setlength{\parskip}{.3cm}

\begin{document}

\author{Robert Crim \\ Department of Aeronautics}
\date{16 September 2011}
\title{Numerical methods for deforming images
\thanks{Submitted to Imperial College London in fulfilment of the requirements for the Degree of Master of Science in Advanced Computational Methods for Aeronautics, Flow Management and Fluid-Structure Interaction}}

\maketitle
\newpage
\begin{abstract}
TODO: write the abstract.
\end{abstract}
\newpage
\tableofcontents
\newpage


\section{Introduction}


 Computational anatomy is a discipline which aims to analyze biological shapes to determine variability across populates, ages, species, and in the presence of disease \cite{miller2009emerging}. It is hoped that the ability to correlate this variability with other anatomical information will aid in the early diagnosis of disease and in surgical planning. (CITE MICCAI 2011 proceedings?)

WHY THIS IS DIFFICULT..

Our approach is inner metric, the other is outer metric

Current approaches the problem include LDDMM, and EPDIFF  \cite{cotter2006singular, cotter2009curves} which involve finding a diffeomorphism with transforms the space to match the template to the target. 

The current report uses a new approach introduced in \cite{bauer2011new}, where the deforming vector is defined directly on the curve itself, rather than the entire field. An inner metric is defined to determine a path between the template and target, which when minimized, resulting in the geodesic between the two shapes.

A benefit of using inner metric spaces is the shape is allowed to intersect itself!

We end up with a constrained optimisation problem

Parameterisation invariant, and why it's important.


\section{Spaces}


$H^1(\Omega)$ is a Sobolev space, containing a test function $v$ such that $v^2$ and $||\nabla v||^2$ have finite integrals in $\Omega$. The Sobolev space allows functions to have discontinuous derivatives.. 


We must use continuous space for our finite elements, since we have to be able to integrate the adjoint, $\hvect q$, and stuff



\section{Mesh and mapping}
The $\Omega$ is an interval from $0 \leq s \leq 2\pi$, in the FEniCS solver a finite element mesh of $[0,2\pi]$ is divided into $M$ elements.

The 1 dimensional interval is mapped to a parametric curve, IS THIS EVEN TRUE? Really need to understand this!



\section{Derivation}
\begin{align*}
  \vect q(s) &= 0 \leq s \leq 2\pi \\
  \vect q(0) &= \vect q(2\pi)
\end{align*}


The template $\vect q_A$ is the curve at $t=0$, and the target $\vect q_B$
is the curve at $t=1$. The scheme finds a velocity $\vect u(s)$ to minimize
distance between the two curves. A BFGS algorithm is used to find the minimal
functional $S[u]$, *REWRITE*

a metric, such that the dq/ds ensures is parameterisation invariant -- and non zero!

Need to explain the metric part, and penalty parts of \eq{S}.

\begin{equation}
  \label{eq:S}
  S = \frac{1}{2} \int^{1}_{t=0} \int^{2\pi}_{s=0}\left( \left| \vect u(s,t) \right|^2 
  \left| \frac{\partial \vect q}{\partial s} \right|  + 
  \alpha^2 \frac{ 
    \left| \frac{\partial \vect u}{\partial s}\right|^2}{
    \left| \frac{\partial \vect q}{\partial s}\right|}\right)  \,ds\,dt
  + \frac{1}{2\sigma^2}\int^{2\pi}_{s=0}\left| q(s,1) - q_B(s)\right|^2\,ds.
\end{equation}
Subject to
\begin{equation}
  \vect q_A(s) =\vect q(s,0)   \label{eq:template}
\end{equation}
and with the dynamical constraint
\begin{equation}
  \frac{\partial \vect q}{\partial t}(s,t) = \vect u(s,t)  \label{eq:dqdt}
\end{equation}
where $\alpha$ is dependent on the mesh size ????, and $\sigma$ is the weighing
of penalty term $\left| q(s,1) - q_B(s)\right|^2$.

Minimising $S[u]$ is easier with it's variational derivative known, such that
\begin{equation}
  \label{eq:lim_dS}
  \int^1_0 \vv{S}{\vect u}\cdot \delta \vect u\,dt = 
  \lim_{\eps \rightarrow 0} \frac{S[\vect u + \eps \delta \vect u] - S[\vect u]}{\eps},
\end{equation}
 however it's difficult to find $\vv{S}{\vect u}$ directly. Since we can find $\vv{S}{\vect q}$, a Lagrange multiplier can be used find $\vv{S}{\vect u}$.

Relaxing the constraint \eq{dqdt} and introducing the Lagrange multiplier $\hvect q$, $S[\vect u]$ can be reformulated as
\begin{equation}
  \label{eq:J}
  J[\vect q, \vect u, \hvect q] = S[\vect u] + \int^1_0\left\langle \hvect q ,\pp{q}{t} - \vect u\right\rangle \,dt
\end{equation}
with the bra-ket notion for inner product over the interval $[0,2\pi]$ with respect to $s$,
\begin{equation}
  \left\langle f(s),g(s) \right\rangle := \int^{2\pi}_0 f(s) \cdot g(s)\,ds.
\end{equation}

If $\vect u$ is chosen to satisfy \eq{dqdt}, then
\begin{equation}
  \label{eq:JeqS}
  J[\vect u, \vect q, \hvect q] \equiv S[\vect u].
\end{equation}
By setting $\pp{J}{\hvect q} = 0$, \eq{dqdt} is satisfied and the derivative of \eq{JeqS} is
\begin{equation}
  \label{eq:dSeqdJ}
  \pp{S}{\vect u} =  \pp{J}{\vect u} + \pp{J}{\vect q}\cdot\pp{\vect q}{\vect u}
\end{equation}

We first find an equation for  $\vv{J}{q}$ using the definition of the functional derivative \eq{lim_dS}
\begin{equation}
  \label{eq:lim_dJdq}
  \left\langle \vv{J}{\vect q}, \delta \vect q \right\rangle =
  \lim_{\eps \rightarrow 0} \frac{J[\vect u, \vect q + \eps \delta \vect q, \hvect q] - J[\vect u, \vect q, \hvect q]}{\eps}
\end{equation}
Expanding the right side of \eq{lim_dJdq} and taking only $O(\eps)$ terms gives
\begin{align}
  \label{eq:dJdq}
  \left\langle \vv{J}{\vect q}, \delta \vect q \right\rangle =&
  \frac{1}{2}\int^1_0\int^{2\pi}_0 |\vect u|^2
  \frac{\pp{}{s}\delta \vect q \cdot \pp{\vect q}{s}}{|\pp{\vect q}{s}|}
    + \alpha^2 \frac{|\pp{\vect u}{s}|^2}{|\pp{\vect q}{s}|^3} \pp{}{s}\delta \vect q \cdot \pp{\vect q}{s} \,ds\,dt  \nonumber \\
    &+ \left\langle \delta \vect q, \vect q\big|_{t=1} - \vect q^B \right\rangle + \int^1_0\left\langle \hvect q, \pp{}{t} \delta \vect q \right\rangle\,dt
\end{align}
Using the property of adjoint equations $\pp{}{t}\angles{x,y} = \angles{\dot x, y} + \angles{x, \dot y}$, we can write
\begin{align}
  \label{eq:adjoint}
  \int^1_0\angles{\hvect q, \delta \pp{}{s} \vect q}\,dt &=  
  \int^1_0\angles{\frac{d}{dt}\hvect q, \delta \vect q}\,dt -  \int^1_0\angles{\pp{\hvect q}{t}, \delta \vect q }\,dt \nonumber \\
  &= \int^1_0\angles{\frac{d}{dt}\hvect q, \delta \vect q}\,dt - \Big[\angles{\hvect q, \delta \vect q}\Big]^1_0 \nonumber \\
  &= \int^1_0\angles{\frac{d}{dt}\hvect q, \delta \vect q}\,dt - \angles{\hvect q\big|_{t=1}, \delta \vect q\big|_{t=0}},
\end{align}
where $\delta \vect q\big|_{t=0} = 0$, due to \eq{template}.

Substituting \eq{adjoint}, \eq{dJdq} can be rewritten as
\begin{align}
  \label{eq:dJdq-adjoint}
  \left\langle \vv{J}{\vect q}, \delta \vect q \right\rangle =&
  \frac{1}{2}\int^1_0\int^{2\pi}_0 -\frac{\hvect q}{t}\cdot \delta \vect q + |\vect u|^2
  \frac{\pp{}{s}\delta \vect q \cdot \pp{\vect q}{s}}{|\pp{\vect q}{s}|}
    + \alpha^2 \frac{|\pp{\vect u}{s}|^2}{|\pp{\vect q}{s}|^3} \pp{}{s}\delta \vect q \cdot \pp{\vect q}{s} \,ds\,dt  \nonumber \\
    &+ \angles{ \hvect q \big|_{t=1} + \frac{1}{\sigma^2}(\vect q \big|_{t=1} - \vect q_B), \delta \vect q}
\end{align}


Since $\vv{J}{\vect q} = 0$, we can separate \eq{dJdq-adjoint} into weak form equations for $\pp{\hvect q}{t}$ and $\hvect q \big|_{t=1}$
  \begin{equation}
    \label{eq:qh_t1}
    \angles{\delta \vect q, \hvect q\big|_{t=1}} = \frac{1}{\sigma^2}\angles{\delta \vect q, \vect q \big|_{t=1} - \vect q_B}
  \end{equation}
  \begin{equation}
    \label{eq:dqhdt}
    \angles{\delta \vect q, \pp{\hvect q}{t}} = \angles{\frac{1}{2}\left(\frac{\left|\vect u\right|}{\left|\pp{\vect u}{s}\right|} + \alpha^2\frac{\left|\pp{\vect u}{s}\right|^2}{\left|\pp{\vect u}{s}\right|^3}\right)\pp{}{s}\delta \vect q, \pp{\vect q}{s}}
  \end{equation}
Finally, we can find a weak form equation for $\vv{S}{\vect u}$,
\begin{equation}
  \label{eq:var_dSdu-dJdu}
  \angles{\vv{S}{\vect u}, \delta \vect u} = \angles{\vv{J}{\vect u}, \delta \vect u} =
  \lim_{\eps \rightarrow 0} \frac{J[\vect u + \eps \delta \vect u, \vect q, \hvect q] - J[\vect u, \vect q, \hvect q]}{\eps}.
\end{equation}
Expanding the right and side of \eq{var_dSdu-dJdu}, and taking only $O(\eps)$ terms yields
\begin{equation}
  \label{eq:dSdu-deltau}
  \angles{\vv{S}{\vect u}, \delta \vect u} = \angles{\vect u \left|\pp{\vect q}{s}\right|,\delta \vect u} 
  + \alpha^2 \angles{\frac{\pp{\vect u}{s}}{\left|\pp{\vect u}{s}\right|},\pp{}{s}\delta \vect u,}
    - \angles{\hvect q,\delta \vect u}.
\end{equation}


\section{Numerical scheme}

An overview of the solution process.

The optimisation agorithm takes an initial guess of $U$, which is a matrix of $\vect u$ vectors at each timestep, and the functions to calculate $S[U]$ and $\vv{S}{U}$. 
\subsection{Discretisation}

write about the eqns FEniCS uses

\subsection{Implementation in FEniCS}
Notes about how FEniCS is used.. Be sure to cite the FEnICS project!

... The equations are easy to put into the form $a(u,v) = L(v)$, where $a(u,v)$ is bilinear, and $L(v)$ is linear translate well to the variational problem formulation used in FEnICS.

How the JIT compiler is used, and how why it's good.

Note how reshaping is needed to work with the minimisation algorithm..


\subsection{L-BFGS minimisation algorithm}


Explain how this works, limited memory BGFS. 

Cite book \cite{nocedal1999numerical} for general BFGS
Cite L-BFGS paper \cite{byrd1995limited}
Cite L-BFGS-B paper \cite{zhu1997algorithm}

\section{Numerical experiments}

Results of small deformations
Results of large deformation

Show U, how U changes

Some way to demonstrate evolution, plot TEN time steps on top?
Show the ``mean'' between two shapes $t = N/2*\delta t$.

Show the convergence test results, try first and second order..
How does it change we use other finite elements? Instead of interval?

\subsection{Small deformations}
  Match shapes with small deformations.


\subsection{Large deformations}
Need test shapes.

\subsection{Shape matching}

Given a curve, we attempt to identity the curve as a particular shape from a small database of known shapes. 

Using a hand drawn star, the scheme was able to successfully identity the shape as a star by deforming into each of the shapes in the database. The smallest $S$ gives the shape which most closely matches the given shape. 

% shape_match/circle.txt 14832680.5754
% shape_match/hexagon.txt 23762865.2721
% shape_match/pentagon.txt 10049668.5673
% shape_match/square.txt 39626911.0532
% shape_match/star.txt 1662214.64395
% shape_match/star_11.txt 14837679.2763
% shape_match/star_six.txt 20461001.086
% shape_match/triangle.txt 27393192.9801
% Matching shape is:  shape_match/star.txt


A limitation with the current code is that scaling and translation is not accounted for.


\section{Conclusion}

A numerical scheme to finding an approximate geodesic between two shapes is presented. Unlike outer metric methods, which deform the space itself, we use inner metrics to attach a deforming vector to the shape itself.

How can code performance be improved? Finding $\sigma$?

Maybe we can try to find an equation for the geodesic using inner metrics, like EPDIFF does on outer metrics. How does the scheme scale? Is the problem suited to very large problems?

% Given a velocity $\vect u(s,t)$, the scheme first finds the vector $\vect q$ at
% each time step from $t=0$ to $t=1$. \eq{dqdt} is discretized using a forward finite difference
% \begin{equation}
%   \label{eq:fd_q}
%   \frac{\vect q(s)^{n+1}- \vect q(s)^n}{\Delta t} = \vect u(s).
% \end{equation}
% Using a test function $\vect r(s)$, \eq{df_q_test} is solved for $\vect q^{n+1}$
% at each time step.
%   \begin{equation}
%     \label{eq:df_q_test}
%     \int \vect{r}(s) \cdot \vect{q}^{n+1}(s)\, ds = \int \vect{r}(s) \cdot
%     \vect{q}^n (s)\,ds + \Delta t \int \vect{r}(s)\cdot \vect{v}(s)\, ds
%   \end{equation}



% The variational derivative $\frac{\delta S}{\delta \vect u}$ is need to improve
% the performance of the optimiser. 

% For a single timestep, the variational derivative can be found with the equation
% \begin{equation}
%   \label{eq:dsdu}
%   \langle \vect v, \frac{\delta S}{\delta \vect u}\rangle =
%   \langle \vect v, \vect u \left| \frac{\partial \vect q}{\partial s}\right |\rangle 
%   + \alpha^2 \langle \frac{\partial \vect v}{\partial s},
%   \frac{\frac{\partial \vect v}{\partial s} }{ \left|\frac{\partial \vect q}{\partial s}\right|}
% \rangle - \langle \vect v, \hvect{q} \rangle,
% \end{equation}
% where $\vect v$ is a test function, and $\hvect q$ is a Lagrange multiplier (SHOW..)

% For each time step, $\hvect q$ is found using backwards finite difference method
% where $\hvect q(s,1)$ is found by
% \begin{equation}
%   \label{eq:qh1}
%   \langle \hvect p, \hvect q(s,1) \rangle = - \frac{1}{\sigma^2}
%   \langle \hvect p, \vect q(s,1) - q_B \rangle
% \end{equation}
% where $\hvect p$ is a test function.


% \begin{equation}
%   \label{eq:dqhdt}
%   \left\langle \hvect{p},\pp{\hvect{q}}{t}\right\rangle = \left\langle \frac{1}{2}\left(
%       \frac{|\vect u|^2}{\left|\pp{\vect q}{s}\right|}
%       -\alpha^2\frac{\left|\pp{\vect u}{s}\right|^2}{\left|\pp{\vect q}{s}\right|^3}\right)
%     \pp{\hvect{p}}{s},\pp{\hvect{q}}{s}\right\rangle 
% \end{equation}

  % OLD---------
  % \langle \hvect p, \frac{\partial \hvect q}{\partial t} \rangle =
  % \langle
  % \left( 
  %   \frac{1}{2}
  %   \frac{ 
  %     \left| \vect u \right|^2}{
  %     \left| \frac{\partial \vect q}{\partial s}\right|}
  %   +
  %   \frac{\alpha^2}{2}
  %   \frac{ 
  %     \left| \frac{\partial \vect u}{\partial s}\right|^2}{
  %     \left| \frac{\partial \vect q}{\partial s}\right|^2}
  % \right)\frac{\partial \hvect p}{\partial s}, \frac{\partial \hvect q}{\partial s} \rangle
  % END OLD----------

% Substituting $\frac{\partial \hvect q}{\partial t}$ with a backwards finite difference and
% rearranging \eq{dqhdt},
% \begin{equation}
%   \label{eq:fd_qh}
%   \int \hvect p \cdot \hvect q^n \,ds = \int \hvect p \cdot \hvect q^{n+1} \,ds
%   - \int
%    \left( 
%       \frac{|\vect u|^2}{\left|\pp{\vect q}{s}\right|}
%       -\alpha^2\frac{\left|\pp{\vect u}{s}\right|^2}{\left|\pp{\vect q}{s}\right|^3}
%   \right) \frac{ \partial \hvect p}{\partial s} 
%   \cdot \frac{\partial \hvect q^{n+1}}{\partial s}\,ds
% \end{equation}
% is solved for $q^n$ at each timestep.






\newpage
\addcontentsline{toc}{section}{References}
\bibliographystyle{unsrt}
\bibliography{refs}{}
\end{document}