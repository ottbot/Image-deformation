\usepackage{amsmath}
\usepackage{amssymb}
\usepackage{vector}
\usepackage{cite}
\usepackage{graphicx}
\usepackage{subfig}
\usepackage[usenames, dvipsnames]{color}

\usepackage{setspace}
\usepackage{fancyhdr}
\usepackage{ifthen}
\usepackage{ifpdf}
\usepackage{float} 

\usepackage[english]{babel}

%-->
%--> Google.com search "hyperref options"
%--> 
%--> http://www.ai.mit.edu/lab/sysadmin/latex/documentation/latex/hyperref/manual.pdf
%--> http://www.chemie.unibas.ch/~vogtp/LaTeX2PDFLaTeX.pdf 
%--> http://www.uni-giessen.de/partosch/eurotex99/ oberdiek/print/sli4a4col.pdf
%--> http://me.in-berlin.de/~miwie/tex-refs/html/latex-packages.html
%-->
\usepackage[ pdftex, plainpages = false, pdfpagelabels, 
             pdfpagelayout = useoutlines,
             bookmarks,
             bookmarksopen = true,
             bookmarksnumbered = true,
             breaklinks = true,
             linktocpage,
             pagebackref,
             colorlinks = false,
             linkcolor = blue,
             urlcolor  = blue,
             citecolor = red,
             anchorcolor = green,
             hyperindex = true,
             hyperfigures
             ]{hyperref} 
\usepackage{graphicx}
%\pdfcompresslevel=9
\DeclareGraphicsExtensions{.png, .jpg, .pdf}


\pdfpageheight=297mm
\pdfpagewidth=210mm

\setlength{\hoffset}{0.00cm}
\setlength{\voffset}{0.00cm}

%\setlength{\evensidemargin}{1.96cm}
%\setlength{\oddsidemargin}{-0.54cm}
%% \setlength{\topmargin}{1mm}
%% \setlength{\headheight}{1.36cm}
%% \setlength{\headsep}{1.00cm}
%% \setlength{\textheight}{20.84cm}
%% \setlength{\textwidth}{14.5cm}
%% \setlength{\marginparsep}{1mm}
%% \setlength{\marginparwidth}{3cm}
%% \setlength{\footskip}{2.36cm}


% \pagestyle{fancy}
% \renewcommand{\sectionmark}[1]{\markright{#1}{}}
% \fancyhf{}
% \fancyhead[RO]{\bfseries\rightmark}
% \fancyhead[LE]{\bfseries\leftmark}
% \fancyfoot[C]{\thepage}
% \renewcommand{\headrulewidth}{0.5pt}
% \renewcommand{\footrulewidth}{0pt}
% \addtolength{\headheight}{0.5pt}
% \fancypagestyle{plain}{
%   \fancyhead{}
%   \renewcommand{\headrulewidth}{0pt}
% }



% The year and term the degree will be officially conferred
\def\degreedate#1{\gdef\@degreedate{#1}}
% The full (unabbreviated) name of the degree
\def\degree#1{\gdef\@degree{#1}}
% The name of your college or department(eg. Trinity, Pembroke, Maths, Physics)
\def\collegeordept#1{\gdef\@collegeordept{#1}}
% The name of your University
\def\university#1{\gdef\@university{#1}}
% Defining the crest
\def\crest#1{\gdef\@crest{#1}}


%define title page layout
\renewcommand{\maketitle}{%
    \renewcommand{\footnotesize}{\small}
    \renewcommand{\footnoterule}{\relax}
    \thispagestyle{empty}
%  \null\vfill
  \begin{center}
    { \Huge {\bfseries {Numerical Methods \\for Deforming Images}} \par}
{\large \vspace*{25mm} {\includegraphics[width=35mm]{img/imperial_crest} \par} \vspace*{22mm}}
    {{\Large \bfseries{\href{mailto:rjcrim@gmail.com}{Robert Crim}}} \par}
{\large \vspace*{1ex}
    {\href{http://www.ic.ac.uk/aeronautics}{Department of Aeronautics} \par}
\vspace*{1ex}
    {\href{http://www.ic.ac.uk}{Imperial College London} \par}
\vspace*{20mm}
    {\small Submitted in part fulfillment of the requirements for the degree of} \par}
\vspace*{0.5ex}
    {\large \it{Master of Science} \par}
\vspace*{1ex}
    {16 September 2011}
  \end{center}
  \null\vfill
}


\newcommand{\eq}[1]{(\ref{eq:#1})}
\newcommand{\fig}[1]{Fig. \ref{fig:#1}}
\newcommand{\vect}[1]{\ensuremath{\mathbf{#1}}}
\newcommand{\hvect}[1]{\ensuremath{\hat{\vect{#1}}}}

\newcommand{\pp}[2]{\frac{\partial #1}{\partial #2}}
\newcommand{\vv}[2]{\frac{\delta #1}{\delta #2}}
\newcommand{\angles}[1]{\left\langle #1 \right\rangle}
\newcommand{\eps}{\varepsilon}
\setlength{\parskip}{.3cm}


%\usepackage{mathpazo}
%\usepackage[mathpazo]{flexisym}

%\usepackage{breqn}

%\graphicspath{{./img/}}

% \pdfpagewidth=\paperwidth 
% \pdfpageheight=\paperheight
